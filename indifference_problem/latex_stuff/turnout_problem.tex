\documentclass[12pt]{article}
\usepackage[margin=.75in]{geometry}

\usepackage{macros}
\usepackage{amsfonts}
%\usepackage{amsmath}
\usepackage{bbold}
\usepackage{enumitem}
%\usepackage{amsthm}
%\newtheorem{definition}{Definition}

\begin{document}
\setlength{\parindent}{0pt}
\setlength{\parskip}{2pt plus1pt}
\setlength{\baselineskip}{10pt plus1pt}

\subsection*{Indifference Problem Notes}
\textbf{Kit Levy}

\subsection*{7/15 Post-Meeting Notes:}
\jessie{Just FYI I have macros for us}
\kit{test}

\underline{Overview:} Looking at how indifferent voters not showing up to vote affects the expected social welfare of voting rules' selected winning candidate
\newline

$u =$ distribution of voters, $f(u) =$ density function of distribution $u$, $F(u) =$ CDF, $F(\frac{1}{2}) =$ number of votes for candidate $a$, $u' =$  voters who actually show up to vote

People in $u$ have utilities $u_i = (p_a, p_b)$, show up to vote with prob $u_{\text{choice \#}1} - u_{\text{choice \#}2}$ aka strength of preference between the two candidates (eventually maybe $\operatorname{entropy}(u)$)

Probability a person will vote: $u_1 - u_2 = \begin{cases} 
u_1 - (1-u_1) & \text{if } u \geq \frac{1}{2} \\
(1 - u_1) - u_1 & \text{if } u \geq \frac{1}{2}\\
\end{cases} = \begin{cases} 
2u_1 - 1 & \text{if } u \geq \frac{1}{2} \\
1 - 2u_1 & \text{if } u \geq \frac{1}{2}\\
\end{cases}$
\newline \newline

\underline{Problem:} Investigating ``distortion'' $ = \frac{\mathbb{E}[sw(f(u), u)]}{\mathbb{E}[sw(f(u'), u)]}$ for different voting rules

$sw(f(u'), u) =$ total social welfare (from all of $u$) in an election where only $u'$ vote

Distortion $\geq 1 \rightarrow$ people not showing up harmed soc. welfare, distortion $< 1 \rightarrow$ people not showing up improved outcome's sw, find examples?

Eventually try to find bounds on distortion for different voting rules
\newline

\underline{Starting subproblem:} imagine $u$ as drawn from a Beta distribution $u_i \sim \operatorname{Beta}(\alpha, \beta)$ (because can extend to more dimensions with $\operatorname{Dirichlet}(\vec{\delta})$, doesn't necessarily have to be)

Find $\alpha, \beta$ such that $\begin{cases}
F(\frac{1}{2}) \geq \frac{1}{2} & \text{(aka more people in $u$ prefer $a$ over $b$)} \\
\mathop{\mathbb{E}}_{u'}[f(u')] = \text{ B} & \text{(expected winner if only $u'$ vote is B)}
\end{cases}$

$\mathop{\mathbb{E}}_{u'}[f(u')] = b$ equiv to $\int_0^{\frac{1}{2}} (1 - 2u) f(u) du \leq \int_{\frac{1}{2}}^1 (2u - 1)f(u)du$ (ex. votes for A $\leq$ ex. votes for B)

$(1 - 2u) =$ probability you vote if located at $u$, $f(u) =$ prob a person's prefs are located at $u$, $\int_0^{\frac{1}{2}} (1 - 2u) f(u) du =$ expected people who will vote (i.e. are in $u'$) and who prefer A
\newline

Constraints:
\begin{enumerate}[nolistsep]
    \setlength{\itemsep}{0pt}
    \setlength{\parskip}{0pt}
    \item $\frac{1}{B(\alpha, \beta)}\frac{\int_0^{\frac{1}{2}} t^{\alpha-1}(1-t)^{\beta-1}dt}{\int_0^1 t^{\alpha-1}(1-t)^{\beta-1}dt} \leq \frac{1}{2}$
    \item $\int_0^\frac{1}{2}(1-2u)(u^{\alpha-1}(1-u)^{\beta-1})du \leq \int_\frac{1}{2}^1(2u-1)(u^{\alpha-1}(1-u)^{\beta-1})du$
\end{enumerate}

\subsubsection*{To-Do List (starting 7/16):}
\begin{enumerate}
    \item Find subproblem's example of $\alpha, \beta$ for various voting rules
    \begin{enumerate}
        \item Investigate whether to do it analytically or with numerical integration
        \item Try for plurality voting, then IRV, Borda, etc.
        \item Write up examples
    \end{enumerate}
    \item Analyze distortion for examples
    \item Investigate bounds for examples
    \item Find examples where distortion $<1$
\end{enumerate}


\subsubsection*{No distribution $u \sim \operatorname{Beta}(\alpha, \beta)$ exists such that A strictly wins with full turnout and B strictly wins with turnout weighted according to $\lvert 2u - 1 \rvert$:}

\begin{proof}
    \setlength{\baselineskip}{12pt}
    \setlength{\lineskip}{3.5pt}
    \setlength{\lineskiplimit}{2pt}
    Let $f(u) = \frac{1}{B(\alpha, \beta)}u^{\alpha-1}(1-u)^{\beta-1}, u \in [0, 1]$. Let $F(x) = \int_0^x f(u) du$. A wins under full turnout and B wins with weighted turnout $\iff$
\begin{description}[nolistsep]
    \item[\normalfont Constraint 1:] $F(\frac{1}{2}) < \frac{1}{2}$ and
    \item[\normalfont Constraint 2:] $\int_0^\frac{1}{2}(1-2u)f(u)du < \int_\frac{1}{2}^1(2u-1)f(u)du$
\end{description}
\textbf{Fulfilling Constraint 1:} \\
$F(\frac{1}{2}) < \frac{1}{2}$ \\
$\implies \int_0^{\frac{1}{2}} f(u) du > \frac{1}{2}$ \\ 
$\implies \alpha < \beta$. \\
\textbf{Fulfilling Constraint 2:} \\
$\int_0^\frac{1}{2}(1-2u)f(u)du < \int_\frac{1}{2}^1(2u-1)f(u)du$ \\
$ \implies - \int_0^\frac{1}{2}(2u-1)f(u)du < \int_\frac{1}{2}^1(2u-1)f(u)du$ \\
$ \implies 0 < \int_\frac{1}{2}^1(2u-1)f(u)du + \int_0^\frac{1}{2}(2u-1)f(u)du$ \\
$\implies \int_0^1(2u-1)f(u)du > 0$ \\
$\implies 2\int_0^1 uf(u)du - \int_0^1f(u)du > 0$ \\
Because $f(u)$ is a probability density, we know $\int_0^1f(u)du = 1$  and $\int_0^1uf(u)du = \mathbb{E}[u]$. \\
Thus, $2\int_0^1 uf(u)du - \int_0^1f(u)du > 0 \implies 2\mathbb{E}[u] - 1 > 0 \implies \mathbb{E}[u] > \frac{1}{2}$ \\
For $u \sim \operatorname{Beta}(\alpha, \beta)$, $\mathbb{E}[u] = \frac{\alpha}{\alpha + \beta}$. \\
Thus, $\mathbb{E}[u] > \frac{1}{2} \implies \frac{\alpha}{\alpha + \beta} > \frac{1}{2}$ \\
$\implies \alpha > \beta$. \\
Because Constraint 1$\iff \alpha < \beta$ and Constraint 2$\iff \alpha > \beta$, there is no distribution $u \sim \operatorname{Beta}(\alpha, \beta)$ that fulfills both constraints. \\
It follows that there is no distribution $u \sim \operatorname{Beta}(\alpha, \beta)$ that fulfills both constraints for any distribution of votes proportional to $\lvert 2u - 1 \rvert$.
\end{proof}

\subsubsection*{No distribution $u \sim \operatorname{Beta}(\alpha, \beta)$ exists such that A strictly wins with full turnout and B strictly wins with turnout weighted according to any function $w$ such that $w(u)$ is symmetric around $u = \frac{1}{2}$:}
\begin{proof}
    \setlength{\baselineskip}{12pt}
    \setlength{\lineskip}{3.5pt}
    \setlength{\lineskiplimit}{2pt}
    Let $w(u)$ be symmetric around $u = \frac{1}{2}$. \\
    Then, $\int_0^1w(u)f(u)du = \int_0^1w(1 - u)f(u)du$ \\
    For $u \sim \operatorname{Beta}(\alpha, \beta)$ we have:
    \begin{itemize}[nolistsep]
        \item $f(u) > f(1 - u)$ on $u \in [0, \frac{1}{2}) \iff \alpha < \beta$
        \item $f(u) = f(1 - u) \iff \alpha = \beta$
        \item $f(u) < f(1 - u)$ on $u \in [0, \frac{1}{2}) \iff \alpha > \beta$
    \end{itemize}
    So if $f(u)$ has more mass on the left (i.e. $a$ wins the unweighted vote with full turnout, i.e. $F(\frac{1}{2}) > \frac{1}{2}$, i.e. $ \alpha < \beta$), and $w(u)$ is symmetric, then: \\
    We know $w(u) \geq 0$, and $f(u) > f(1 - u)$ on $u \in [0, \frac{1}{2}$, so $w(u)f(u) > w(u)f(1 - u)$ \\
    $\implies \int_0^{\frac{1}{2}}w(u)f(u)du > \int_0^{\frac{1}{2}}w(u)f(1 - u)du$ \\
    Let $t = 1 - u$ (while $u \in [0, \frac{1}{2}]$). $\int_0^{\frac{1}{2}}w(u)f(1 - u)du = \int_1^{\frac{1}{2}}w(1 - t)f(t)(-dt) = \int^1_{\frac{1}{2}}w(t)f(t)dt$ \\
    $\implies \int_0^{\frac{1}{2}}w(u)f(u)du > \int^1_{\frac{1}{2}}w(t)f(t)dt$ \\
    $\implies \int_0^{\frac{1}{2}}w(u)f(u)du > \int^1_{\frac{1}{2}}w(u)f(u)du$ \\
    $\implies$ weighted votes for a $>$ weighted votes for b. \\
    Thus, for votes distributed according to $u \sim \operatorname{Beta}(\alpha, \beta)$, a wins the unweighted election $\implies$ a wins any election weighted according to a function $w$ such that $w(u)$ is symmetric around $u = \frac{1}{2}$.
\end{proof}

\subsection*{7/23 Post-Meeting Notes:}

Distortion = $\frac{sw(a^*, u)}{\mathbb{E}[sw(f(P'), u)]}$, $a^* =$ optimal welfare alternative, $P =$ fixed preference profile, $P' \subseteq P =$ after dropout according to $w(u)$

\underline{Next goals}:
\begin{enumerate}[nolistsep]
    \item Axiomatize/parametrize $w$
    \begin{itemize}[nolistsep]
        \item $w$ should be symmetric around $u = \frac{1}{2}$, need $w(x) = w(\sigma(x))$ (permutation of $x$)
        \item Look at goals of voting, give characteristics of $w$ maybe
        \item Conj: if $w$ is symmetric and single-peaked at $\frac{1}{2}$ then $\mathbb{E}[sw(f(P'), u)] \geq \mathbb{E}[sw(f(P), u)]$
    \end{itemize}
    \item Two "types of voters"/model with two lines of different proportions
    \begin{itemize}[nolistsep]
        \item $v_a : u_a$ and $v_b : u_b$ (two utility locations), $Pr[v_a] = p_a$ and $Pr[v_b] = p_b$ (two proportions of each group)
        \item Find restrictions on $u_a, u_b, p_a$ (lin eq) where $F(\frac{1}{2}) \geq \frac{1}{2}$ and $\int_\frac{1}{2}^1w(x)f(x)dx \geq \int_0^\frac{1}{2}w(x)f(x)dx$
        \item Fix $w(x) = (2x-1)^2 \rightarrow$ parametrize as $(cx-\frac{c}{2})^2$ (different steepnesses of curve)
    \end{itemize}
    \item Characterize continuous (Lipschitz) distributions $f$ where for symmetric $w$ weighted turnout can flip outcome
    \begin{itemize}[nolistsep]
        \item $\{f: F(\frac{1}{2}) \geq \frac{1}{2}, \int_\frac{1}{2}^1w(x)f(x)dx \geq \int_0^\frac{1}{2}w(x)f(x)dx\}$
        \item $\iff \int_\frac{1}{2}^1w(x)[f(x)-f(1-x)]dx \geq 0$, parametrize somehow?
    \end{itemize}
\end{enumerate}

\subsubsection*{Two groups model:}
Majority prefers a and b wins election with turnout $\iff$
\begin{equation*}
  \left\{\begin{array}{@{}l@{}}
    p_a > 1 - p_a \\
    w(u_b)(1 - p_a) > w(u_a)p_a \\
  \end{array}\right.\
\end{equation*}
\begin{align*}
    &w(u_b)(1 - p_a) > w(u_a)p_a \\
    &\implies w(u_b) - p_a(w(u_b) + w(u_a)) > 0 \\
    &\implies \frac{1}{2} < p_a < \frac{w(u_b)}{w(u_b) + w(u_a)}
\end{align*}
For $w(u) = \lvert 2u - 1 \rvert$:
\begin{align*}
    &p_a < \frac{w(u_b)}{w(u_b) + w(u_a)} \\
    &\implies p_a < \frac{2u_b - 1}{2u_b - 1 + 1 - 2u_a} \\
    &\implies p_a < \frac{2u_b - 1}{2u_b - 2u_a} \\
    &\implies \frac{1}{2} < \frac{2u_b - 1}{2u_b - 2u_a} \\
    &\implies u_b - u_a < 2u_b - 1 \\
    &\implies u_a + u_b > 1
\end{align*}

\subsubsection*{Investigating possible $f$:}
Goal: characterize continuous $f$ where weighting turnout according to a symmetric $w$ can flip the outcome of an election, $\{f: F(\frac{1}{2}) \geq \frac{1}{2}, \int_\frac{1}{2}^1w(x)f(x)dx \geq \int_0^\frac{1}{2}w(x)f(x)dx\}$

\underline{Approaching as a convex optimization problem}:

$D(f) = \int_\frac{1}{2}^1 w(u)[f(u) - f(1-u)]du$

Feasible space: $S = \{f \in C^0([0,1]): f \geq 0, \int_0^1f = 1, \int_0^\frac{1}{2}f > \frac{1}{2}\}$

Maximize $D(f)$ over $S$ (aka asking question: $\exists f \in S$ s.t. $D(f) > 0$?)

(see Python script, not sure how to interpret)
\newline

\underline{Trying to linearize everything with nudge function}:

$f(u) = f_0(u) + \epsilon g(u)$ where $f_0(u) = 1$ (uniform density for baseline), $\int_0^1 g(u) du = 0$, $\epsilon < < 1$ controls deviation (still fulfills pdf requirements)

Full turnout condition:

$\int_0^\frac{1}{2} f(u)du = \int_0^\frac{1}{2} (1 + \epsilon g(u)) du = \frac{1}{2} + \epsilon \int_0^\frac{1}{2}g(u)du$

$\implies$ A wins under full turnout $\iff \epsilon \int_0^\frac{1}{2}g(u)du > 0$

Weighted turnout condition:

$\int_\frac{1}{2}^1w(u)f(u)du - \int_0^\frac{1}{2}w(u)f(u)du = \int_0^1w(u)f(u)\operatorname{sgn}(u - \frac{1}{2})du$

$\int_0^1w(u)(1 + \epsilon g(u))\operatorname{sgn}(u - \frac{1}{2})du = \int_0^1w(u)\operatorname{sgn}(u - \frac{1}{2})du + \epsilon \int_0^1w(u)g(u)\operatorname{sgn}(u - \frac{1}{2})du$

$\int_0^1w(u)\operatorname{sgn}(u - \frac{1}{2})du = 0$ because $\operatorname{sgn}(u - \frac{1}{2})$ flips from $-1$ to $1$ and $w$ is symmetric, so the two halves should cancel out.

$\implies$ B wins with weighted turnout $\iff \epsilon\int_0^1w(u)g(u)\operatorname{sgn}(u - \frac{1}{2})du > 0$

A wins under full turnout and B wins with weighted turnout $\iff$
\begin{description}[nolistsep]
    \item[\normalfont Constraint 1:] $\epsilon \int_0^\frac{1}{2}g(u)du > 0$ and
    \item[\normalfont Constraint 2:] $\epsilon\int_0^1w(u)g(u)\operatorname{sgn}(u - \frac{1}{2})du > 0$
\end{description}


\subsection*{7/29 Post-Meeting Notes:}

\underline{To-Do}:

\begin{itemize}[nolistsep]
    \item Conjecture: FOSD of $G_B$ vs. $G_A$ is sufficient to enable swap
        \begin{itemize}[nolistsep]
            \item $g_A: [0, \frac{1}{2}] = f(x), g_B: [0, \frac{1}{2}] = f(1 - x)$ gives strength of prefs, $[0, \frac{1}{2}]$ maps strong to weak preferences 
            \item First order stochastic dominance: point-for-point $G_B \geq G_A$
            \item Sufficient that flip over symmetric $w$ can occur with $u \sim f$
        \end{itemize}
    \item Explore whether FOSD is necessary
    \item Try to write "closest case" of $f$ as function of $w$
        \begin{itemize}[nolistsep]
            \item Closest case minimizes votes for B - votes for A
            \item Look at Python script for examples, $w$ closest case looks quadratic when $w$ is quadratic, etc.
        \end{itemize}
    \item Claim: can characterize $\operatorname{dist}(V, w)$ for majority on two outcomes not achieving optimal
        \begin{itemize}[nolistsep]
            \item Comes from characterizing $f$ from $w$?
            \item $V$ is voting rule, $P$ is total population, $P'$ is weighted voters
        \end{itemize}
\end{itemize}

\subsubsection*{Conjecture: FOSD of $G_B$ vs. $G_A$ is sufficient to enable swap}

\begin{definition}
Let $g, h$ be two probability densities on $\mathbb{R}$ such that $\mathbb{E}_{X \sim g}[\lvert X \rvert], \mathbb{E}_{X \sim h}[\lvert X \rvert]$ are both finite. $g$ has first-order stochastic dominance over $h$ 
\begin{align*}
    &\iff \forall u: \mathbb{R} \to \mathbb{R}\text{ that is non-decreasing, }\mathbb{E}_{X \sim g}[a(X)] \geq \mathbb{E}_{X \sim h}[u(X)] \\
    &\iff \forall t \in \mathbb{R}, G(t) \leq H(t) \\
    &\iff \exists\text{ two random variables }X \sim g, Y \sim h\text{ such that }X = Y + \delta\text{ where }\delta \geq 0
\end{align*}
\end{definition}

    Let $f(x): [0,1]$ be some probability density from which voters' preferences are drawn, with $F(x): [0,1]$ as its cumulative distribution function. Let $g_A: [0, \frac{1}{2}] = f(x), g_B: [0, \frac{1}{2}] = f(1 - x)$. Then, let $G_A = \int_0^xg_A(x)dx =  \int_0^xf(x)dx = F(x)$, and $G_B = \int_0^xg_B(x)dx = \int_0^xf(1 - x)dx = -\int_1^{1-x}f(x)dx = \int_{1-x}^1f(x)dx = F(1) - F(1-x)$.
         
   \jessie{Notes/outline here, hoping to maybe correct something that's been misstated on my end (which is very likely)}
\begin{proof}
Observe that the expected number of votes candidate $a$ receives is $\int_0^{\frac 1 2} w(x) f(x) dx$, and the expected number of votes candidate $b$ receives is 
\begin{align*}
v_B &= \int_{\frac 1 2}^1 w(x) f(x) dx\\
&= \int_{0}^{\frac 1 2} w(1-x) f(1-x) dx \qquad \text{reparameterizing the integral} \\
&= \int_{0}^{\frac 1 2} w(x) f(1-x) dx \qquad \text{symmetry of $w$: $w(x) = w(1-x)$.} \\
&\overset{?}{>} v_A = \int_{0}^{\frac 1 2} w(x) f(x) dx \\
&\overset{?}{\implies} \int_{0}^{\frac 1 2} f(1-x) dx > \int_{0}^{\frac 1 2} f(x) dx \qquad \text{Monotonicity of $w$ \jessie{FALSE; $w$ affine is sufficient for the statement to hold though}}
\end{align*}
\end{proof}

\subsection*{8/8 Post-Meeting Notes:}

\begin{itemize}[nolistsep]
    \item Set $f$ to be class of linear functions as in Python script
    \item $G_A$ must cross $G_B$ at point $c$
    \item Hypothesis: flip happens if $w$ concave, decreasing, $w''(c)=0$
        \begin{itemize}[nolistsep]
            \item Look at integration by parts of $\int f(u) w(u) du$
            \item (Only for linear $f$?)
        \end{itemize}
\end{itemize}




\end{document}

